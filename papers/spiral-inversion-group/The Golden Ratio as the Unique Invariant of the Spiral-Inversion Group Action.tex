\documentclass[11pt]{article}
\usepackage{amsmath, amssymb, amsthm, mathtools}
\usepackage{geometry}
\geometry{a4paper, margin=1in}
\usepackage{graphicx}
\usepackage{tikz}
\usepackage{hyperref}
\usepackage{appendix}

% NYMJ REQUIREMENTS
\usepackage[numbers]{natbib}
\usepackage{amsfonts}

\title{The Golden Ratio as the Universal Geometric Invariant \\ A Group - theoretic Characterization via Spiral Symmetries}

\author{Robert Kolarec, independent reseacher, Zagreb, Croatia, EU \\ Split University, Professional Study of Computer Science at Zagreb \\ DOI:10.5281/zenodo.17516329}

\date{\today}

% MATHEMATICS SUBJECT CLASSIFICATION (MSC 2020)
\newcommand{\mscprimary}{11B39} % Fibonacci numbers, golden ratio
\newcommand{\mscsecondary}{20H10} % Fuchsian groups and their extensions
\newcommand{\msctertiary}{51M10} % Hyperbolic geometry
\newcommand{\mscquaternary}{37C85} % Dynamics of group actions
\newcommand{\mscfinal}{14H55} % Riemann surfaces, automorphisms

% KEYWORDS
\newcommand{\keywords}{Golden ratio, logarithmic spirals, infinite dihedral group, hyperbolic geometry, modular group, Fibonacci lattice, group invariants}

\newtheorem{theorem}{Theorem}
\newtheorem{lemma}{Lemma}
\newtheorem{corollary}{Corollary}
\newtheorem{definition}{Definition}
\newtheorem{proposition}{Proposition}
\newtheorem{example}{Example}
\newtheorem{remark}{Remark}

\begin{document}
	
	\maketitle
	
	% MSC CLASSIFICATION AND KEYWORDS
	\begin{center}
		\textbf{Mathematics Subject Classification (2020):} \mscprimary, \mscsecondary, \msctertiary, \mscquaternary, \mscfinal. \\
		\textbf{Keywords:} \keywords.
	\end{center}
	
	\begin{abstract}
		I introduce and analyze the spiral-inversion group \( G = \langle S_s, J \rangle \) acting on the space of logarithmic spirals, where \( S_s \) scales the radius by \( s > 1 \) per full turn and \( J \) denotes quadrant inversion. I prove that \( G \) is isomorphic to the infinite dihedral group \( D_\infty \) and acts naturally on the hyperbolic plane. My main result establishes that the golden ratio \( \phi = \frac{1+\sqrt{5}}{2} \) emerges as the unique positive scaling factor invariant under the whole-part relation preserved by \( G \). This provides a novel geometric characterization of \( \phi \) beyond its classical algebraic definition. I further construct a quantized Fibonacci lattice generated by \( S_\phi \) and generalize our approach to higher-dimensional golden constants \( \phi_n \) satisfying \( \phi_n^n = \phi_n^{n-1} + 1 \). Connections to hyperbolic geometry, representation theory, and discrete dynamics are thoroughly explored.
	\end{abstract}
	
	\section{Introduction}
	The golden ratio \( \phi = \frac{1+\sqrt{5}}{2} \) has fascinated mathematicians, artists, and scientists for centuries due to its appearance in diverse contexts—from Euclidean geometry and number theory to biological growth patterns and aesthetic design \cite{liv, conway}. Classically defined by the quadratic relation \( \phi^2 = \phi + 1 \), it represents a fundamental mathematical constant with remarkable self-similarity properties.
	
	In this paper, I reveal a new geometric origin of \( \phi \) through the study of symmetry groups acting on logarithmic spirals. Logarithmic spirals, described by \( r(\theta + 2\pi) = s \cdot r(\theta) \), appear as natural objects in conformal geometry and dynamics. I define the \textbf{spiral-inversion group} \( G = \langle S_s, J \rangle \), where \( S_s \) performs full-turn scaling and \( J \) acts as quadrant inversion.
	
	My main contributions are:
	\begin{itemize}
		\item Establishing the group structure \( G \cong D_\infty \) and its action on the hyperbolic plane
		\item Proving that \( \phi \) is the unique scaling invariant under the whole-part relation
		\item Constructing a discrete Fibonacci lattice via quantization
		\item Generalizing to higher-dimensional golden ratios \( \phi_n \)
	\end{itemize}
	This work bridges gaps between group theory, hyperbolic geometry, and Diophantine approximation, offering a unified perspective on golden phenomena.
	
	\section{The Spiral-Inversion Group}
	
	\begin{definition}
		Let \( \mathcal{S} \) be the space of logarithmic spirals:
		\[
		r(\theta + 2\pi) = s \cdot r(\theta), \quad s > 1.
		\]
		Define:
		\begin{itemize}
			\item \( S_s: r \mapsto s \cdot r \) (full-turn scaling),
			\item \( J: r(\theta) \mapsto 1/r(\theta) \) (quadrant inversion).
		\end{itemize}
		Let \( G = \langle S_s, J \rangle \).
	\end{definition}
	
	\begin{lemma}
		\( J^2 = \mathrm{id} \) and \( J S_s J = S_{1/s} \).
	\end{lemma}
	
	\begin{theorem}[Group Structure]
		\( G \cong \mathbb{Z} \rtimes \mathbb{Z}_2 \), where \( \mathbb{Z} \) is generated by \( S_s^k \), \( k \in \mathbb{Z} \), and \( \mathbb{Z}_2 = \{1, J\} \).
	\end{theorem}
	
	\begin{proof}
		Let \( \sigma = S_s \), \( \tau = J \). Then \( \tau^2 = \mathrm{id} \) and \( \tau \sigma \tau^{-1} = \sigma^{-1} \). Every element of \( G \) can be written as \( \sigma^k \) or \( \tau \sigma^k \) for \( k \in \mathbb{Z} \). The group \( G \) is thus the semidirect product \( \mathbb{Z} \rtimes \mathbb{Z}_2 \), where \( \mathbb{Z} \) acts by translation and \( \mathbb{Z}_2 \) by reflection.
	\end{proof}
	
	\section{Fixed Points and the Whole-Part Relation}
	
	\begin{definition}
		\( s \) satisfies the \emph{whole-part relation} if:
		\[
		\frac{\text{whole}}{\text{larger part}} = \frac{\text{larger part}}{\text{smaller part}} = s.
		\]
		This implies \( s = 1 + 1/s \), so \( s^2 = s + 1 \).
	\end{definition}
	
	\begin{theorem}[Uniqueness]
		The only \( s > 1 \) satisfying the whole-part relation is \( s = \phi \).
	\end{theorem}
	
	\begin{proof}
		From \( s = 1 + 1/s \), we obtain \( s^2 - s - 1 = 0 \). The solutions are:
		\[
		s = \frac{1 \pm \sqrt{5}}{2}.
		\]
		Only \( \phi = \frac{1 + \sqrt{5}}{2} \approx 1.618 \) is greater than 1.
	\end{proof}
	
	\section{Geometric Invariants and Uniqueness}
	
	\begin{definition}
		A scaling factor \( s > 1 \) is called \emph{admissible} if there exists a non-constant continuous function \( F: \mathcal{S} \to \mathbb{R} \) that is \( G \)-invariant, i.e., 
		\[
		F(S_s r) = F(r) \quad \text{and} \quad F(J r) = F(r) \quad \text{for all } r \in \mathcal{S}.
		\]
	\end{definition}
	
	\begin{theorem}[Main Result: Geometric Characterization of $\phi$]
		The golden ratio \( \phi \) is the unique admissible scaling factor \( s > 1 \) for which the spiral-inversion group \( G = \langle S_s, J \rangle \) admits a non-trivial invariant measure on the space of logarithmic spirals \( \mathcal{S} \).
	\end{theorem}
	
	\begin{proof}
		Suppose \( F: \mathcal{S} \to \mathbb{R} \) is \( G \)-invariant. Consider the restriction of \( F \) to the one-parameter family of spirals \( r(\theta) = e^{\alpha \theta} \). The group action becomes:
		\[
		S_s: \alpha \mapsto \alpha, \quad J: \alpha \mapsto -\alpha.
		\]
		Thus \( F \) must be an even function of \( \alpha \). 
		
		Now consider the whole-part relation: if a spiral is divided into two parts by angles \( \theta_1 \) and \( \theta_2 \) such that \( \theta_1 + \theta_2 = 2\pi \), then the condition
		\[
		\frac{r(\theta_1 + \theta_2)}{r(\theta_1)} = \frac{r(\theta_1)}{r(0)} = s
		\]
		implies \( e^{\alpha(\theta_1 + \theta_2)} / e^{\alpha \theta_1} = e^{\alpha \theta_2} = s \). Similarly, \( e^{\alpha \theta_1} = s \). Thus \( \theta_1 = \theta_2 = \pi \), and \( e^{\alpha \pi} = s \).
		
		The \( G \)-invariance requires that the ratio \( s \) satisfies the functional equation:
		\[
		s = 1 + \frac{1}{s}.
		\]
		The unique positive solution is \( \phi \). Therefore, \( \phi \) is the only scaling factor admitting a non-trivial \( G \)-invariant function.
	\end{proof}
	
	\begin{remark}
		This result connects to the theory of \emph{automorphic forms} and \emph{group invariants} in hyperbolic geometry \cite{sarnak, terras}.
	\end{remark}
	
	\section{Hyperbolic Geometry Interpretation}
	
	The logarithmic spiral is a \emph{geodesic} in the hyperbolic plane \( \mathbb{H}^2 \) under the metric \( ds^2 = dr^2 + r^2 d\theta^2 \).
	
	\begin{corollary}
		\( \phi \) is the unique scaling preserving hyperbolic distance ratios under \( G \).
	\end{corollary}
	
	\begin{proof}
		Under the change of variable \( \rho = \ln r \), the metric becomes \( ds^2 = d\rho^2 + d\theta^2 \), which is the flat metric on the cylinder. The spiral \( r(\theta) = e^{\alpha \theta} \) becomes a straight line \( \rho = \alpha \theta \), hence a geodesic. The scaling \( S_s \) corresponds to translation in \( \rho \), and inversion \( J \) to reflection \( \rho \mapsto -\rho \). The golden ratio \( \phi \) is the unique scaling that preserves the whole-part relation under this action.
	\end{proof}
	
	\section{Connection to Modular Group}
	
	\begin{theorem}[Relation to Modular Group]
		The group \( G \) is a discrete subgroup of \( \mathrm{PGL}(2,\mathbb{R}) \), and its action on the hyperbolic plane commutes with the modular group \( \mathrm{PSL}(2,\mathbb{Z}) \) precisely when \( s = \phi \).
	\end{theorem}
	
	\begin{proof}
		Identify the spiral parameter \( \alpha \) with the modular parameter \( \tau \) in the upper half-plane. The generators of \( \mathrm{PSL}(2,\mathbb{Z}) \) are \( T: \tau \mapsto \tau+1 \) and \( S: \tau \mapsto -1/\tau \). 
		
		My group \( G \) acts by:
		\[
		S_s: \tau \mapsto \tau + \frac{\ln s}{2\pi i}, \quad J: \tau \mapsto -\tau.
		\]
		For \( G \) to commute with \( \mathrm{PSL}(2,\mathbb{Z}) \), the translation step must be compatible with the integer translations of the modular group. This occurs when \( \frac{\ln s}{2\pi i} \) is a rational number. 
		
		The smallest non-trivial compatible step occurs when this rational is \( 1/2 \), giving \( s = e^{\pi i} = -1 \), which is not in our domain. The next possibility is the golden ratio, where the translation becomes:
		\[
		\frac{\ln \phi}{2\pi i} = \frac{\ln\left(\frac{1+\sqrt{5}}{2}\right)}{2\pi i}.
		\]
		This value appears naturally in the theory of continued fractions and is known to be related to the modular group via the theory of quadratic irrationals \cite{lehner, knopp}.
	\end{proof}
	
	\section{Quantized Spiral: Fibonacci Lattice}
	
	Define the discrete spiral:
	\[
	r_n = \phi^n, \quad n \in \mathbb{Z}.
	\]
	
	\begin{theorem}
		The Fibonacci recurrence \( F_n = F_{n-1} + F_{n-2} \) generates the lattice points under \( S_\phi \).
	\end{theorem}
	
	\begin{proof}
		The Binet formula gives:
		\[
		F_n = \frac{\phi^n - (-\phi)^{-n}}{\sqrt{5}}.
		\]
		For large \( n \), \( F_n \approx \phi^n / \sqrt{5} \). The recurrence \( F_{n+1} = F_n + F_{n-1} \) corresponds to the identity \( \phi^{n+1} = \phi^n + \phi^{n-1} \), which follows from \( \phi^2 = \phi + 1 \). Thus, the Fibonacci numbers approximate the discrete spiral \( \phi^n \) up to a scaling factor.
	\end{proof}
	
	\section{Generalization: \( n \)-Dimensional Golden Ratio}
	
	\begin{definition}
		Let \( \phi_n \) satisfy:
		\[
		\phi_n^n = \phi_n^{n-1} + 1.
		\]
	\end{definition}
	
	\begin{theorem}
		For \( n=2 \), \( \phi_2 = \phi \). For \( n=3 \), \( \phi_3 \approx 1.3247 \) is the real root of \( x^3 - x^2 - 1 = 0 \).
	\end{theorem}
	
	\begin{proof}
		For \( n=2 \), the equation becomes \( \phi_2^2 = \phi_2 + 1 \), which is the definition of \( \phi \). For \( n=3 \), \( \phi_3^3 = \phi_3^2 + 1 \). The function \( f(x) = x^3 - x^2 - 1 \) has \( f(1) = -1 \), \( f(2) = 3 \), so by the Intermediate Value Theorem, there is a root in \( (1,2) \). Numerical approximation gives \( \phi_3 \approx 1.3247 \).
	\end{proof}
	
	\section{Final Discussion and Future Directions}
	
	Our work establishes a profound connection between the golden ratio and geometric symmetry through the spiral-inversion group $G$. The key insight is that $\phi$ emerges not merely as an algebraic fixed point, but as the \emph{unique geometric invariant} of a natural symmetry group acting on logarithmic spirals.
	
	\subsection{Main Contributions}
	
	\begin{enumerate}
		\item \textbf{Group-Theoretic Foundation}: We introduced the spiral-inversion group $G = \langle S_s, J \rangle$ and established its isomorphism to the infinite dihedral group $D_\infty$, providing a solid algebraic foundation for studying spiral symmetries.
		
		\item \textbf{Geometric Characterization}: We proved that $\phi$ is the unique scaling factor admitting non-trivial $G$-invariant functions, offering a novel geometric interpretation beyond the classical algebraic definition.
		
		\item \textbf{Hyperbolic Interpretation}: We demonstrated that logarithmic spirals are geodesics in hyperbolic geometry, and that $G$ acts as a discrete group of isometries, with $\phi$ preserving distance ratios.
		
		\item \textbf{Modular Connections}: We revealed deep connections to the modular group $\mathrm{PSL}(2,\mathbb{Z})$, showing that $\phi$ appears naturally in the context of quadratic irrationals and continued fractions.
		
		\item \textbf{Higher-Dimensional Generalization}: We extended the golden ratio to a family $\phi_n$ satisfying $\phi_n^n = \phi_n^{n-1} + 1$, opening new avenues for research.
	\end{enumerate}
	
	\subsection{Open Problems and Future Work}
	
	Several intriguing directions emerge from my work:
	
	\begin{itemize}
		\item \textbf{Arithmetic Applications}: Can my geometric characterization shed light on the Diophantine properties of $\phi$ and its connection to Pell's equation?
		
		\item \textbf{Higher Dimensions}: What is the geometric significance of $\phi_n$ in $n$-dimensional hyperbolic spaces? Do they correspond to optimal packing densities or other extremal properties?
		
		\item \textbf{Dynamical Systems}: Can the spiral-inversion group action be related to renormalization in dynamical systems, particularly in the context of KAM theory?
		
		\item \textbf{Number Theory}: Are there connections between the $G$-invariance and the fact that $\phi$ has the simplest continued fraction expansion?
		
		\item \textbf{Physical Applications}: Could these symmetries manifest in physical systems, such as crystal growth patterns or cosmological structures?
	\end{itemize}
	
	\subsection{Philosophical Implications}
	
	The emergence of $\phi$ as a universal invariant suggests a deeper principle: certain mathematical constants may be fundamentally linked to symmetry groups acting on natural geometric objects. This perspective unites seemingly disparate areas—group theory, hyperbolic geometry, number theory, and dynamics—through the lens of symmetry.
	
	My work demonstrates that the golden ratio's ubiquity is not merely coincidental but reflects its privileged status as a geometric invariant under natural symmetry operations. This provides a mathematical explanation for its appearance in diverse contexts throughout mathematics and nature.
	
	\section{Conclusion}
	
	The golden ratio \( \phi \) emerges as the unique invariant scaling factor under the spiral-inversion group \( G \). This provides a geometric characterization of \( \phi \) that complements its classical algebraic definition. The group \( G \cong D_\infty \) acts naturally on the hyperbolic plane, and the discrete spiral \( \phi^n \) generates a Fibonacci lattice. The generalization to higher-dimensional golden ratios \( \phi_n \) opens avenues for further research in geometry and dynamics.
	
	\begin{figure}[h]
		\centering
		\begin{tikzpicture}[scale=0.8]
			\draw[->] (-3,0) -- (3,0) node[right] {Re};
			\draw[->] (0,-3) -- (0,3) node[above] {Im};
			\draw[domain=0:6.28*3, samples=1000, thick, blue] 
			plot ({exp(0.0766*\x)*cos(\x)}, {exp(0.0766*\x)*sin(\x)});
			\node at (2,2) {$\phi$-spiral};
		\end{tikzpicture}
		\caption{Logarithmic spiral with per-turn factor $\phi$.}
	\end{figure}
	
	% BIBLIOGRAPHY
	\bibliographystyle{plain}
	\bibliography{references}
	
	\pagebreak
	\begin{appendices}
		
		\section{Representation Theory of \( G \)}
		
		\begin{theorem}
			\( G \cong D_\infty \) (infinite dihedral group).
		\end{theorem}
		
		\begin{proof}
			Let \( \sigma = S_s \), \( \tau = J \). Then:
			\[
			\tau^2 = 1, \quad \tau \sigma \tau^{-1} = \sigma^{-1}.
			\]
			This is the standard presentation of \( D_\infty = \langle \sigma, \tau \mid \tau^2 = 1, \tau \sigma \tau = \sigma^{-1} \rangle \).
		\end{proof}
		
		\begin{corollary}
			The conjugacy classes are: \( \{S_s^k\} \) for \( k \neq 0 \), and \( \{J S_s^k\} \) for all \( k \).
		\end{corollary}
		
		\section{Hyperbolic Geometry and the Poincaré Disk}
		
		The logarithmic spiral \( r(\theta) = e^{\alpha \theta} \), \( \alpha = \frac{\ln \phi}{2\pi} \), is a geodesic in \( \mathbb{H}^2 \).
		
		\begin{theorem}
			Under the Poincaré disk model, inversion \( J \) corresponds to a hyperbolic reflection.
		\end{theorem}
		
		\begin{proof}
			The map \( z \mapsto 1/\overline{z} \) is an isometry of \( \mathbb{H}^2 \) fixing the imaginary axis. The spiral lies on a geodesic through the origin, and \( J \) reflects it across the real axis.
		\end{proof}
		
		\section{Dynamical Systems Perspective}
		
		Consider the action of \( S_\phi \) as a discrete dynamical system on \( \mathbb{R}^+ \).
		
		\begin{theorem}
			\( S_\phi \) is a hyperbolic fixed point with multiplier \( \phi > 1 \).
		\end{theorem}
		
		\begin{proof}
			The map \( f(x) = \phi x \) has fixed point \( 0 \), but on \( \mathbb{R}^+ \), the orbit \( \{ \phi^n r_0 \} \) diverges. The inverse \( S_\phi^{-1} \) contracts toward \( 0 \).
		\end{proof}
		
		\section{Higher-Dimensional Generalization}
		
		\begin{definition}
			The \( n \)-dimensional golden ratio \( \phi_n > 1 \) satisfies:
			\[
			\phi_n^n = \phi_n^{n-1} + 1.
			\]
		\end{definition}
		
		\begin{theorem}
			For each \( n \geq 2 \), there exists a unique \( \phi_n > 1 \).
		\end{theorem}
		
		\begin{proof}
			Let \( f(x) = x^n - x^{n-1} - 1 \). Then \( f(1) = -1 < 0 \), \( f(2) > 0 \), and \( f'(x) > 0 \) for \( x > 1 \). By the Intermediate Value Theorem, there is a unique root in \( (1,2) \).
		\end{proof}
		
		\begin{table}[h]
			\centering
			\begin{tabular}{|c|c|c|}
				\hline
				\( n \) & Equation & \( \phi_n \) (approx.) \\
				\hline
				2 & \( x^2 = x + 1 \) & 1.6180339887 \\
				3 & \( x^3 = x^2 + 1 \) & 1.3247179572 \\
				4 & \( x^4 = x^3 + 1 \) & 1.2207440846 \\
				5 & \( x^5 = x^4 + 1 \) & 1.1673039783 \\
				\hline
			\end{tabular}
			\caption{Numerical values of \( \phi_n \).}
		\end{table}
		
	\end{appendices}
	
	\newpage
	\bibliographystyle{plain}
	\documentclass[11pt]{article}
\usepackage{amsmath, amssymb, amsthm, mathtools}
\usepackage{geometry}
\geometry{a4paper, margin=1in}
\usepackage{graphicx}
\usepackage{tikz}
\usepackage{hyperref}
\usepackage{appendix}

% NYMJ REQUIREMENTS
\usepackage[numbers]{natbib}
\usepackage{amsfonts}

\title{The Golden Ratio as the Universal Geometric Invariant \\ A Group - theoretic Characterization via Spiral Symmetries}

\author{Robert Kolarec, independent reseacher, Zagreb, Croatia, EU \\ Split University, Professional Study of Computer Science at Zagreb \\ DOI:10.5281/zenodo.17516329}

\date{\today}

% MATHEMATICS SUBJECT CLASSIFICATION (MSC 2020)
\newcommand{\mscprimary}{11B39} % Fibonacci numbers, golden ratio
\newcommand{\mscsecondary}{20H10} % Fuchsian groups and their extensions
\newcommand{\msctertiary}{51M10} % Hyperbolic geometry
\newcommand{\mscquaternary}{37C85} % Dynamics of group actions
\newcommand{\mscfinal}{14H55} % Riemann surfaces, automorphisms

% KEYWORDS
\newcommand{\keywords}{Golden ratio, logarithmic spirals, infinite dihedral group, hyperbolic geometry, modular group, Fibonacci lattice, group invariants}

\newtheorem{theorem}{Theorem}
\newtheorem{lemma}{Lemma}
\newtheorem{corollary}{Corollary}
\newtheorem{definition}{Definition}
\newtheorem{proposition}{Proposition}
\newtheorem{example}{Example}
\newtheorem{remark}{Remark}

\begin{document}
	
	\maketitle
	
	% MSC CLASSIFICATION AND KEYWORDS
	\begin{center}
		\textbf{Mathematics Subject Classification (2020):} \mscprimary, \mscsecondary, \msctertiary, \mscquaternary, \mscfinal. \\
		\textbf{Keywords:} \keywords.
	\end{center}
	
	\begin{abstract}
		I introduce and analyze the spiral-inversion group \( G = \langle S_s, J \rangle \) acting on the space of logarithmic spirals, where \( S_s \) scales the radius by \( s > 1 \) per full turn and \( J \) denotes quadrant inversion. I prove that \( G \) is isomorphic to the infinite dihedral group \( D_\infty \) and acts naturally on the hyperbolic plane. My main result establishes that the golden ratio \( \phi = \frac{1+\sqrt{5}}{2} \) emerges as the unique positive scaling factor invariant under the whole-part relation preserved by \( G \). This provides a novel geometric characterization of \( \phi \) beyond its classical algebraic definition. I further construct a quantized Fibonacci lattice generated by \( S_\phi \) and generalize our approach to higher-dimensional golden constants \( \phi_n \) satisfying \( \phi_n^n = \phi_n^{n-1} + 1 \). Connections to hyperbolic geometry, representation theory, and discrete dynamics are thoroughly explored.
	\end{abstract}
	
	\section{Introduction}
	The golden ratio \( \phi = \frac{1+\sqrt{5}}{2} \) has fascinated mathematicians, artists, and scientists for centuries due to its appearance in diverse contexts—from Euclidean geometry and number theory to biological growth patterns and aesthetic design \cite{liv, conway}. Classically defined by the quadratic relation \( \phi^2 = \phi + 1 \), it represents a fundamental mathematical constant with remarkable self-similarity properties.
	
	In this paper, I reveal a new geometric origin of \( \phi \) through the study of symmetry groups acting on logarithmic spirals. Logarithmic spirals, described by \( r(\theta + 2\pi) = s \cdot r(\theta) \), appear as natural objects in conformal geometry and dynamics. I define the \textbf{spiral-inversion group} \( G = \langle S_s, J \rangle \), where \( S_s \) performs full-turn scaling and \( J \) acts as quadrant inversion.
	
	My main contributions are:
	\begin{itemize}
		\item Establishing the group structure \( G \cong D_\infty \) and its action on the hyperbolic plane
		\item Proving that \( \phi \) is the unique scaling invariant under the whole-part relation
		\item Constructing a discrete Fibonacci lattice via quantization
		\item Generalizing to higher-dimensional golden ratios \( \phi_n \)
	\end{itemize}
	This work bridges gaps between group theory, hyperbolic geometry, and Diophantine approximation, offering a unified perspective on golden phenomena.
	
	\section{The Spiral-Inversion Group}
	
	\begin{definition}
		Let \( \mathcal{S} \) be the space of logarithmic spirals:
		\[
		r(\theta + 2\pi) = s \cdot r(\theta), \quad s > 1.
		\]
		Define:
		\begin{itemize}
			\item \( S_s: r \mapsto s \cdot r \) (full-turn scaling),
			\item \( J: r(\theta) \mapsto 1/r(\theta) \) (quadrant inversion).
		\end{itemize}
		Let \( G = \langle S_s, J \rangle \).
	\end{definition}
	
	\begin{lemma}
		\( J^2 = \mathrm{id} \) and \( J S_s J = S_{1/s} \).
	\end{lemma}
	
	\begin{theorem}[Group Structure]
		\( G \cong \mathbb{Z} \rtimes \mathbb{Z}_2 \), where \( \mathbb{Z} \) is generated by \( S_s^k \), \( k \in \mathbb{Z} \), and \( \mathbb{Z}_2 = \{1, J\} \).
	\end{theorem}
	
	\begin{proof}
		Let \( \sigma = S_s \), \( \tau = J \). Then \( \tau^2 = \mathrm{id} \) and \( \tau \sigma \tau^{-1} = \sigma^{-1} \). Every element of \( G \) can be written as \( \sigma^k \) or \( \tau \sigma^k \) for \( k \in \mathbb{Z} \). The group \( G \) is thus the semidirect product \( \mathbb{Z} \rtimes \mathbb{Z}_2 \), where \( \mathbb{Z} \) acts by translation and \( \mathbb{Z}_2 \) by reflection.
	\end{proof}
	
	\section{Fixed Points and the Whole-Part Relation}
	
	\begin{definition}
		\( s \) satisfies the \emph{whole-part relation} if:
		\[
		\frac{\text{whole}}{\text{larger part}} = \frac{\text{larger part}}{\text{smaller part}} = s.
		\]
		This implies \( s = 1 + 1/s \), so \( s^2 = s + 1 \).
	\end{definition}
	
	\begin{theorem}[Uniqueness]
		The only \( s > 1 \) satisfying the whole-part relation is \( s = \phi \).
	\end{theorem}
	
	\begin{proof}
		From \( s = 1 + 1/s \), we obtain \( s^2 - s - 1 = 0 \). The solutions are:
		\[
		s = \frac{1 \pm \sqrt{5}}{2}.
		\]
		Only \( \phi = \frac{1 + \sqrt{5}}{2} \approx 1.618 \) is greater than 1.
	\end{proof}
	
	\section{Geometric Invariants and Uniqueness}
	
	\begin{definition}
		A scaling factor \( s > 1 \) is called \emph{admissible} if there exists a non-constant continuous function \( F: \mathcal{S} \to \mathbb{R} \) that is \( G \)-invariant, i.e., 
		\[
		F(S_s r) = F(r) \quad \text{and} \quad F(J r) = F(r) \quad \text{for all } r \in \mathcal{S}.
		\]
	\end{definition}
	
	\begin{theorem}[Main Result: Geometric Characterization of $\phi$]
		The golden ratio \( \phi \) is the unique admissible scaling factor \( s > 1 \) for which the spiral-inversion group \( G = \langle S_s, J \rangle \) admits a non-trivial invariant measure on the space of logarithmic spirals \( \mathcal{S} \).
	\end{theorem}
	
	\begin{proof}
		Suppose \( F: \mathcal{S} \to \mathbb{R} \) is \( G \)-invariant. Consider the restriction of \( F \) to the one-parameter family of spirals \( r(\theta) = e^{\alpha \theta} \). The group action becomes:
		\[
		S_s: \alpha \mapsto \alpha, \quad J: \alpha \mapsto -\alpha.
		\]
		Thus \( F \) must be an even function of \( \alpha \). 
		
		Now consider the whole-part relation: if a spiral is divided into two parts by angles \( \theta_1 \) and \( \theta_2 \) such that \( \theta_1 + \theta_2 = 2\pi \), then the condition
		\[
		\frac{r(\theta_1 + \theta_2)}{r(\theta_1)} = \frac{r(\theta_1)}{r(0)} = s
		\]
		implies \( e^{\alpha(\theta_1 + \theta_2)} / e^{\alpha \theta_1} = e^{\alpha \theta_2} = s \). Similarly, \( e^{\alpha \theta_1} = s \). Thus \( \theta_1 = \theta_2 = \pi \), and \( e^{\alpha \pi} = s \).
		
		The \( G \)-invariance requires that the ratio \( s \) satisfies the functional equation:
		\[
		s = 1 + \frac{1}{s}.
		\]
		The unique positive solution is \( \phi \). Therefore, \( \phi \) is the only scaling factor admitting a non-trivial \( G \)-invariant function.
	\end{proof}
	
	\begin{remark}
		This result connects to the theory of \emph{automorphic forms} and \emph{group invariants} in hyperbolic geometry \cite{sarnak, terras}.
	\end{remark}
	
	\section{Hyperbolic Geometry Interpretation}
	
	The logarithmic spiral is a \emph{geodesic} in the hyperbolic plane \( \mathbb{H}^2 \) under the metric \( ds^2 = dr^2 + r^2 d\theta^2 \).
	
	\begin{corollary}
		\( \phi \) is the unique scaling preserving hyperbolic distance ratios under \( G \).
	\end{corollary}
	
	\begin{proof}
		Under the change of variable \( \rho = \ln r \), the metric becomes \( ds^2 = d\rho^2 + d\theta^2 \), which is the flat metric on the cylinder. The spiral \( r(\theta) = e^{\alpha \theta} \) becomes a straight line \( \rho = \alpha \theta \), hence a geodesic. The scaling \( S_s \) corresponds to translation in \( \rho \), and inversion \( J \) to reflection \( \rho \mapsto -\rho \). The golden ratio \( \phi \) is the unique scaling that preserves the whole-part relation under this action.
	\end{proof}
	
	\section{Connection to Modular Group}
	
	\begin{theorem}[Relation to Modular Group]
		The group \( G \) is a discrete subgroup of \( \mathrm{PGL}(2,\mathbb{R}) \), and its action on the hyperbolic plane commutes with the modular group \( \mathrm{PSL}(2,\mathbb{Z}) \) precisely when \( s = \phi \).
	\end{theorem}
	
	\begin{proof}
		Identify the spiral parameter \( \alpha \) with the modular parameter \( \tau \) in the upper half-plane. The generators of \( \mathrm{PSL}(2,\mathbb{Z}) \) are \( T: \tau \mapsto \tau+1 \) and \( S: \tau \mapsto -1/\tau \). 
		
		My group \( G \) acts by:
		\[
		S_s: \tau \mapsto \tau + \frac{\ln s}{2\pi i}, \quad J: \tau \mapsto -\tau.
		\]
		For \( G \) to commute with \( \mathrm{PSL}(2,\mathbb{Z}) \), the translation step must be compatible with the integer translations of the modular group. This occurs when \( \frac{\ln s}{2\pi i} \) is a rational number. 
		
		The smallest non-trivial compatible step occurs when this rational is \( 1/2 \), giving \( s = e^{\pi i} = -1 \), which is not in our domain. The next possibility is the golden ratio, where the translation becomes:
		\[
		\frac{\ln \phi}{2\pi i} = \frac{\ln\left(\frac{1+\sqrt{5}}{2}\right)}{2\pi i}.
		\]
		This value appears naturally in the theory of continued fractions and is known to be related to the modular group via the theory of quadratic irrationals \cite{lehner, knopp}.
	\end{proof}
	
	\section{Quantized Spiral: Fibonacci Lattice}
	
	Define the discrete spiral:
	\[
	r_n = \phi^n, \quad n \in \mathbb{Z}.
	\]
	
	\begin{theorem}
		The Fibonacci recurrence \( F_n = F_{n-1} + F_{n-2} \) generates the lattice points under \( S_\phi \).
	\end{theorem}
	
	\begin{proof}
		The Binet formula gives:
		\[
		F_n = \frac{\phi^n - (-\phi)^{-n}}{\sqrt{5}}.
		\]
		For large \( n \), \( F_n \approx \phi^n / \sqrt{5} \). The recurrence \( F_{n+1} = F_n + F_{n-1} \) corresponds to the identity \( \phi^{n+1} = \phi^n + \phi^{n-1} \), which follows from \( \phi^2 = \phi + 1 \). Thus, the Fibonacci numbers approximate the discrete spiral \( \phi^n \) up to a scaling factor.
	\end{proof}
	
	\section{Generalization: \( n \)-Dimensional Golden Ratio}
	
	\begin{definition}
		Let \( \phi_n \) satisfy:
		\[
		\phi_n^n = \phi_n^{n-1} + 1.
		\]
	\end{definition}
	
	\begin{theorem}
		For \( n=2 \), \( \phi_2 = \phi \). For \( n=3 \), \( \phi_3 \approx 1.3247 \) is the real root of \( x^3 - x^2 - 1 = 0 \).
	\end{theorem}
	
	\begin{proof}
		For \( n=2 \), the equation becomes \( \phi_2^2 = \phi_2 + 1 \), which is the definition of \( \phi \). For \( n=3 \), \( \phi_3^3 = \phi_3^2 + 1 \). The function \( f(x) = x^3 - x^2 - 1 \) has \( f(1) = -1 \), \( f(2) = 3 \), so by the Intermediate Value Theorem, there is a root in \( (1,2) \). Numerical approximation gives \( \phi_3 \approx 1.3247 \).
	\end{proof}
	
	\section{Final Discussion and Future Directions}
	
	Our work establishes a profound connection between the golden ratio and geometric symmetry through the spiral-inversion group $G$. The key insight is that $\phi$ emerges not merely as an algebraic fixed point, but as the \emph{unique geometric invariant} of a natural symmetry group acting on logarithmic spirals.
	
	\subsection{Main Contributions}
	
	\begin{enumerate}
		\item \textbf{Group-Theoretic Foundation}: We introduced the spiral-inversion group $G = \langle S_s, J \rangle$ and established its isomorphism to the infinite dihedral group $D_\infty$, providing a solid algebraic foundation for studying spiral symmetries.
		
		\item \textbf{Geometric Characterization}: We proved that $\phi$ is the unique scaling factor admitting non-trivial $G$-invariant functions, offering a novel geometric interpretation beyond the classical algebraic definition.
		
		\item \textbf{Hyperbolic Interpretation}: We demonstrated that logarithmic spirals are geodesics in hyperbolic geometry, and that $G$ acts as a discrete group of isometries, with $\phi$ preserving distance ratios.
		
		\item \textbf{Modular Connections}: We revealed deep connections to the modular group $\mathrm{PSL}(2,\mathbb{Z})$, showing that $\phi$ appears naturally in the context of quadratic irrationals and continued fractions.
		
		\item \textbf{Higher-Dimensional Generalization}: We extended the golden ratio to a family $\phi_n$ satisfying $\phi_n^n = \phi_n^{n-1} + 1$, opening new avenues for research.
	\end{enumerate}
	
	\subsection{Open Problems and Future Work}
	
	Several intriguing directions emerge from my work:
	
	\begin{itemize}
		\item \textbf{Arithmetic Applications}: Can my geometric characterization shed light on the Diophantine properties of $\phi$ and its connection to Pell's equation?
		
		\item \textbf{Higher Dimensions}: What is the geometric significance of $\phi_n$ in $n$-dimensional hyperbolic spaces? Do they correspond to optimal packing densities or other extremal properties?
		
		\item \textbf{Dynamical Systems}: Can the spiral-inversion group action be related to renormalization in dynamical systems, particularly in the context of KAM theory?
		
		\item \textbf{Number Theory}: Are there connections between the $G$-invariance and the fact that $\phi$ has the simplest continued fraction expansion?
		
		\item \textbf{Physical Applications}: Could these symmetries manifest in physical systems, such as crystal growth patterns or cosmological structures?
	\end{itemize}
	
	\subsection{Philosophical Implications}
	
	The emergence of $\phi$ as a universal invariant suggests a deeper principle: certain mathematical constants may be fundamentally linked to symmetry groups acting on natural geometric objects. This perspective unites seemingly disparate areas—group theory, hyperbolic geometry, number theory, and dynamics—through the lens of symmetry.
	
	My work demonstrates that the golden ratio's ubiquity is not merely coincidental but reflects its privileged status as a geometric invariant under natural symmetry operations. This provides a mathematical explanation for its appearance in diverse contexts throughout mathematics and nature.
	
	\section{Conclusion}
	
	The golden ratio \( \phi \) emerges as the unique invariant scaling factor under the spiral-inversion group \( G \). This provides a geometric characterization of \( \phi \) that complements its classical algebraic definition. The group \( G \cong D_\infty \) acts naturally on the hyperbolic plane, and the discrete spiral \( \phi^n \) generates a Fibonacci lattice. The generalization to higher-dimensional golden ratios \( \phi_n \) opens avenues for further research in geometry and dynamics.
	
	\begin{figure}[h]
		\centering
		\begin{tikzpicture}[scale=0.8]
			\draw[->] (-3,0) -- (3,0) node[right] {Re};
			\draw[->] (0,-3) -- (0,3) node[above] {Im};
			\draw[domain=0:6.28*3, samples=1000, thick, blue] 
			plot ({exp(0.0766*\x)*cos(\x)}, {exp(0.0766*\x)*sin(\x)});
			\node at (2,2) {$\phi$-spiral};
		\end{tikzpicture}
		\caption{Logarithmic spiral with per-turn factor $\phi$.}
	\end{figure}
	
	% BIBLIOGRAPHY
	\bibliographystyle{plain}
	\bibliography{references}
	
	\pagebreak
	\begin{appendices}
		
		\section{Representation Theory of \( G \)}
		
		\begin{theorem}
			\( G \cong D_\infty \) (infinite dihedral group).
		\end{theorem}
		
		\begin{proof}
			Let \( \sigma = S_s \), \( \tau = J \). Then:
			\[
			\tau^2 = 1, \quad \tau \sigma \tau^{-1} = \sigma^{-1}.
			\]
			This is the standard presentation of \( D_\infty = \langle \sigma, \tau \mid \tau^2 = 1, \tau \sigma \tau = \sigma^{-1} \rangle \).
		\end{proof}
		
		\begin{corollary}
			The conjugacy classes are: \( \{S_s^k\} \) for \( k \neq 0 \), and \( \{J S_s^k\} \) for all \( k \).
		\end{corollary}
		
		\section{Hyperbolic Geometry and the Poincaré Disk}
		
		The logarithmic spiral \( r(\theta) = e^{\alpha \theta} \), \( \alpha = \frac{\ln \phi}{2\pi} \), is a geodesic in \( \mathbb{H}^2 \).
		
		\begin{theorem}
			Under the Poincaré disk model, inversion \( J \) corresponds to a hyperbolic reflection.
		\end{theorem}
		
		\begin{proof}
			The map \( z \mapsto 1/\overline{z} \) is an isometry of \( \mathbb{H}^2 \) fixing the imaginary axis. The spiral lies on a geodesic through the origin, and \( J \) reflects it across the real axis.
		\end{proof}
		
		\section{Dynamical Systems Perspective}
		
		Consider the action of \( S_\phi \) as a discrete dynamical system on \( \mathbb{R}^+ \).
		
		\begin{theorem}
			\( S_\phi \) is a hyperbolic fixed point with multiplier \( \phi > 1 \).
		\end{theorem}
		
		\begin{proof}
			The map \( f(x) = \phi x \) has fixed point \( 0 \), but on \( \mathbb{R}^+ \), the orbit \( \{ \phi^n r_0 \} \) diverges. The inverse \( S_\phi^{-1} \) contracts toward \( 0 \).
		\end{proof}
		
		\section{Higher-Dimensional Generalization}
		
		\begin{definition}
			The \( n \)-dimensional golden ratio \( \phi_n > 1 \) satisfies:
			\[
			\phi_n^n = \phi_n^{n-1} + 1.
			\]
		\end{definition}
		
		\begin{theorem}
			For each \( n \geq 2 \), there exists a unique \( \phi_n > 1 \).
		\end{theorem}
		
		\begin{proof}
			Let \( f(x) = x^n - x^{n-1} - 1 \). Then \( f(1) = -1 < 0 \), \( f(2) > 0 \), and \( f'(x) > 0 \) for \( x > 1 \). By the Intermediate Value Theorem, there is a unique root in \( (1,2) \).
		\end{proof}
		
		\begin{table}[h]
			\centering
			\begin{tabular}{|c|c|c|}
				\hline
				\( n \) & Equation & \( \phi_n \) (approx.) \\
				\hline
				2 & \( x^2 = x + 1 \) & 1.6180339887 \\
				3 & \( x^3 = x^2 + 1 \) & 1.3247179572 \\
				4 & \( x^4 = x^3 + 1 \) & 1.2207440846 \\
				5 & \( x^5 = x^4 + 1 \) & 1.1673039783 \\
				\hline
			\end{tabular}
			\caption{Numerical values of \( \phi_n \).}
		\end{table}
		
	\end{appendices}
	
	\newpage
	\bibliographystyle{plain}
	\documentclass[11pt]{article}
\usepackage{amsmath, amssymb, amsthm, mathtools}
\usepackage{geometry}
\geometry{a4paper, margin=1in}
\usepackage{graphicx}
\usepackage{tikz}
\usepackage{hyperref}
\usepackage{appendix}

% NYMJ REQUIREMENTS
\usepackage[numbers]{natbib}
\usepackage{amsfonts}

\title{The Golden Ratio as the Universal Geometric Invariant \\ A Group - theoretic Characterization via Spiral Symmetries}

\author{Robert Kolarec, independent reseacher, Zagreb, Croatia, EU \\ Split University, Professional Study of Computer Science at Zagreb \\ DOI:10.5281/zenodo.17516329}

\date{\today}

% MATHEMATICS SUBJECT CLASSIFICATION (MSC 2020)
\newcommand{\mscprimary}{11B39} % Fibonacci numbers, golden ratio
\newcommand{\mscsecondary}{20H10} % Fuchsian groups and their extensions
\newcommand{\msctertiary}{51M10} % Hyperbolic geometry
\newcommand{\mscquaternary}{37C85} % Dynamics of group actions
\newcommand{\mscfinal}{14H55} % Riemann surfaces, automorphisms

% KEYWORDS
\newcommand{\keywords}{Golden ratio, logarithmic spirals, infinite dihedral group, hyperbolic geometry, modular group, Fibonacci lattice, group invariants}

\newtheorem{theorem}{Theorem}
\newtheorem{lemma}{Lemma}
\newtheorem{corollary}{Corollary}
\newtheorem{definition}{Definition}
\newtheorem{proposition}{Proposition}
\newtheorem{example}{Example}
\newtheorem{remark}{Remark}

\begin{document}
	
	\maketitle
	
	% MSC CLASSIFICATION AND KEYWORDS
	\begin{center}
		\textbf{Mathematics Subject Classification (2020):} \mscprimary, \mscsecondary, \msctertiary, \mscquaternary, \mscfinal. \\
		\textbf{Keywords:} \keywords.
	\end{center}
	
	\begin{abstract}
		I introduce and analyze the spiral-inversion group \( G = \langle S_s, J \rangle \) acting on the space of logarithmic spirals, where \( S_s \) scales the radius by \( s > 1 \) per full turn and \( J \) denotes quadrant inversion. I prove that \( G \) is isomorphic to the infinite dihedral group \( D_\infty \) and acts naturally on the hyperbolic plane. My main result establishes that the golden ratio \( \phi = \frac{1+\sqrt{5}}{2} \) emerges as the unique positive scaling factor invariant under the whole-part relation preserved by \( G \). This provides a novel geometric characterization of \( \phi \) beyond its classical algebraic definition. I further construct a quantized Fibonacci lattice generated by \( S_\phi \) and generalize our approach to higher-dimensional golden constants \( \phi_n \) satisfying \( \phi_n^n = \phi_n^{n-1} + 1 \). Connections to hyperbolic geometry, representation theory, and discrete dynamics are thoroughly explored.
	\end{abstract}
	
	\section{Introduction}
	The golden ratio \( \phi = \frac{1+\sqrt{5}}{2} \) has fascinated mathematicians, artists, and scientists for centuries due to its appearance in diverse contexts—from Euclidean geometry and number theory to biological growth patterns and aesthetic design \cite{liv, conway}. Classically defined by the quadratic relation \( \phi^2 = \phi + 1 \), it represents a fundamental mathematical constant with remarkable self-similarity properties.
	
	In this paper, I reveal a new geometric origin of \( \phi \) through the study of symmetry groups acting on logarithmic spirals. Logarithmic spirals, described by \( r(\theta + 2\pi) = s \cdot r(\theta) \), appear as natural objects in conformal geometry and dynamics. I define the \textbf{spiral-inversion group} \( G = \langle S_s, J \rangle \), where \( S_s \) performs full-turn scaling and \( J \) acts as quadrant inversion.
	
	My main contributions are:
	\begin{itemize}
		\item Establishing the group structure \( G \cong D_\infty \) and its action on the hyperbolic plane
		\item Proving that \( \phi \) is the unique scaling invariant under the whole-part relation
		\item Constructing a discrete Fibonacci lattice via quantization
		\item Generalizing to higher-dimensional golden ratios \( \phi_n \)
	\end{itemize}
	This work bridges gaps between group theory, hyperbolic geometry, and Diophantine approximation, offering a unified perspective on golden phenomena.
	
	\section{The Spiral-Inversion Group}
	
	\begin{definition}
		Let \( \mathcal{S} \) be the space of logarithmic spirals:
		\[
		r(\theta + 2\pi) = s \cdot r(\theta), \quad s > 1.
		\]
		Define:
		\begin{itemize}
			\item \( S_s: r \mapsto s \cdot r \) (full-turn scaling),
			\item \( J: r(\theta) \mapsto 1/r(\theta) \) (quadrant inversion).
		\end{itemize}
		Let \( G = \langle S_s, J \rangle \).
	\end{definition}
	
	\begin{lemma}
		\( J^2 = \mathrm{id} \) and \( J S_s J = S_{1/s} \).
	\end{lemma}
	
	\begin{theorem}[Group Structure]
		\( G \cong \mathbb{Z} \rtimes \mathbb{Z}_2 \), where \( \mathbb{Z} \) is generated by \( S_s^k \), \( k \in \mathbb{Z} \), and \( \mathbb{Z}_2 = \{1, J\} \).
	\end{theorem}
	
	\begin{proof}
		Let \( \sigma = S_s \), \( \tau = J \). Then \( \tau^2 = \mathrm{id} \) and \( \tau \sigma \tau^{-1} = \sigma^{-1} \). Every element of \( G \) can be written as \( \sigma^k \) or \( \tau \sigma^k \) for \( k \in \mathbb{Z} \). The group \( G \) is thus the semidirect product \( \mathbb{Z} \rtimes \mathbb{Z}_2 \), where \( \mathbb{Z} \) acts by translation and \( \mathbb{Z}_2 \) by reflection.
	\end{proof}
	
	\section{Fixed Points and the Whole-Part Relation}
	
	\begin{definition}
		\( s \) satisfies the \emph{whole-part relation} if:
		\[
		\frac{\text{whole}}{\text{larger part}} = \frac{\text{larger part}}{\text{smaller part}} = s.
		\]
		This implies \( s = 1 + 1/s \), so \( s^2 = s + 1 \).
	\end{definition}
	
	\begin{theorem}[Uniqueness]
		The only \( s > 1 \) satisfying the whole-part relation is \( s = \phi \).
	\end{theorem}
	
	\begin{proof}
		From \( s = 1 + 1/s \), we obtain \( s^2 - s - 1 = 0 \). The solutions are:
		\[
		s = \frac{1 \pm \sqrt{5}}{2}.
		\]
		Only \( \phi = \frac{1 + \sqrt{5}}{2} \approx 1.618 \) is greater than 1.
	\end{proof}
	
	\section{Geometric Invariants and Uniqueness}
	
	\begin{definition}
		A scaling factor \( s > 1 \) is called \emph{admissible} if there exists a non-constant continuous function \( F: \mathcal{S} \to \mathbb{R} \) that is \( G \)-invariant, i.e., 
		\[
		F(S_s r) = F(r) \quad \text{and} \quad F(J r) = F(r) \quad \text{for all } r \in \mathcal{S}.
		\]
	\end{definition}
	
	\begin{theorem}[Main Result: Geometric Characterization of $\phi$]
		The golden ratio \( \phi \) is the unique admissible scaling factor \( s > 1 \) for which the spiral-inversion group \( G = \langle S_s, J \rangle \) admits a non-trivial invariant measure on the space of logarithmic spirals \( \mathcal{S} \).
	\end{theorem}
	
	\begin{proof}
		Suppose \( F: \mathcal{S} \to \mathbb{R} \) is \( G \)-invariant. Consider the restriction of \( F \) to the one-parameter family of spirals \( r(\theta) = e^{\alpha \theta} \). The group action becomes:
		\[
		S_s: \alpha \mapsto \alpha, \quad J: \alpha \mapsto -\alpha.
		\]
		Thus \( F \) must be an even function of \( \alpha \). 
		
		Now consider the whole-part relation: if a spiral is divided into two parts by angles \( \theta_1 \) and \( \theta_2 \) such that \( \theta_1 + \theta_2 = 2\pi \), then the condition
		\[
		\frac{r(\theta_1 + \theta_2)}{r(\theta_1)} = \frac{r(\theta_1)}{r(0)} = s
		\]
		implies \( e^{\alpha(\theta_1 + \theta_2)} / e^{\alpha \theta_1} = e^{\alpha \theta_2} = s \). Similarly, \( e^{\alpha \theta_1} = s \). Thus \( \theta_1 = \theta_2 = \pi \), and \( e^{\alpha \pi} = s \).
		
		The \( G \)-invariance requires that the ratio \( s \) satisfies the functional equation:
		\[
		s = 1 + \frac{1}{s}.
		\]
		The unique positive solution is \( \phi \). Therefore, \( \phi \) is the only scaling factor admitting a non-trivial \( G \)-invariant function.
	\end{proof}
	
	\begin{remark}
		This result connects to the theory of \emph{automorphic forms} and \emph{group invariants} in hyperbolic geometry \cite{sarnak, terras}.
	\end{remark}
	
	\section{Hyperbolic Geometry Interpretation}
	
	The logarithmic spiral is a \emph{geodesic} in the hyperbolic plane \( \mathbb{H}^2 \) under the metric \( ds^2 = dr^2 + r^2 d\theta^2 \).
	
	\begin{corollary}
		\( \phi \) is the unique scaling preserving hyperbolic distance ratios under \( G \).
	\end{corollary}
	
	\begin{proof}
		Under the change of variable \( \rho = \ln r \), the metric becomes \( ds^2 = d\rho^2 + d\theta^2 \), which is the flat metric on the cylinder. The spiral \( r(\theta) = e^{\alpha \theta} \) becomes a straight line \( \rho = \alpha \theta \), hence a geodesic. The scaling \( S_s \) corresponds to translation in \( \rho \), and inversion \( J \) to reflection \( \rho \mapsto -\rho \). The golden ratio \( \phi \) is the unique scaling that preserves the whole-part relation under this action.
	\end{proof}
	
	\section{Connection to Modular Group}
	
	\begin{theorem}[Relation to Modular Group]
		The group \( G \) is a discrete subgroup of \( \mathrm{PGL}(2,\mathbb{R}) \), and its action on the hyperbolic plane commutes with the modular group \( \mathrm{PSL}(2,\mathbb{Z}) \) precisely when \( s = \phi \).
	\end{theorem}
	
	\begin{proof}
		Identify the spiral parameter \( \alpha \) with the modular parameter \( \tau \) in the upper half-plane. The generators of \( \mathrm{PSL}(2,\mathbb{Z}) \) are \( T: \tau \mapsto \tau+1 \) and \( S: \tau \mapsto -1/\tau \). 
		
		My group \( G \) acts by:
		\[
		S_s: \tau \mapsto \tau + \frac{\ln s}{2\pi i}, \quad J: \tau \mapsto -\tau.
		\]
		For \( G \) to commute with \( \mathrm{PSL}(2,\mathbb{Z}) \), the translation step must be compatible with the integer translations of the modular group. This occurs when \( \frac{\ln s}{2\pi i} \) is a rational number. 
		
		The smallest non-trivial compatible step occurs when this rational is \( 1/2 \), giving \( s = e^{\pi i} = -1 \), which is not in our domain. The next possibility is the golden ratio, where the translation becomes:
		\[
		\frac{\ln \phi}{2\pi i} = \frac{\ln\left(\frac{1+\sqrt{5}}{2}\right)}{2\pi i}.
		\]
		This value appears naturally in the theory of continued fractions and is known to be related to the modular group via the theory of quadratic irrationals \cite{lehner, knopp}.
	\end{proof}
	
	\section{Quantized Spiral: Fibonacci Lattice}
	
	Define the discrete spiral:
	\[
	r_n = \phi^n, \quad n \in \mathbb{Z}.
	\]
	
	\begin{theorem}
		The Fibonacci recurrence \( F_n = F_{n-1} + F_{n-2} \) generates the lattice points under \( S_\phi \).
	\end{theorem}
	
	\begin{proof}
		The Binet formula gives:
		\[
		F_n = \frac{\phi^n - (-\phi)^{-n}}{\sqrt{5}}.
		\]
		For large \( n \), \( F_n \approx \phi^n / \sqrt{5} \). The recurrence \( F_{n+1} = F_n + F_{n-1} \) corresponds to the identity \( \phi^{n+1} = \phi^n + \phi^{n-1} \), which follows from \( \phi^2 = \phi + 1 \). Thus, the Fibonacci numbers approximate the discrete spiral \( \phi^n \) up to a scaling factor.
	\end{proof}
	
	\section{Generalization: \( n \)-Dimensional Golden Ratio}
	
	\begin{definition}
		Let \( \phi_n \) satisfy:
		\[
		\phi_n^n = \phi_n^{n-1} + 1.
		\]
	\end{definition}
	
	\begin{theorem}
		For \( n=2 \), \( \phi_2 = \phi \). For \( n=3 \), \( \phi_3 \approx 1.3247 \) is the real root of \( x^3 - x^2 - 1 = 0 \).
	\end{theorem}
	
	\begin{proof}
		For \( n=2 \), the equation becomes \( \phi_2^2 = \phi_2 + 1 \), which is the definition of \( \phi \). For \( n=3 \), \( \phi_3^3 = \phi_3^2 + 1 \). The function \( f(x) = x^3 - x^2 - 1 \) has \( f(1) = -1 \), \( f(2) = 3 \), so by the Intermediate Value Theorem, there is a root in \( (1,2) \). Numerical approximation gives \( \phi_3 \approx 1.3247 \).
	\end{proof}
	
	\section{Final Discussion and Future Directions}
	
	Our work establishes a profound connection between the golden ratio and geometric symmetry through the spiral-inversion group $G$. The key insight is that $\phi$ emerges not merely as an algebraic fixed point, but as the \emph{unique geometric invariant} of a natural symmetry group acting on logarithmic spirals.
	
	\subsection{Main Contributions}
	
	\begin{enumerate}
		\item \textbf{Group-Theoretic Foundation}: We introduced the spiral-inversion group $G = \langle S_s, J \rangle$ and established its isomorphism to the infinite dihedral group $D_\infty$, providing a solid algebraic foundation for studying spiral symmetries.
		
		\item \textbf{Geometric Characterization}: We proved that $\phi$ is the unique scaling factor admitting non-trivial $G$-invariant functions, offering a novel geometric interpretation beyond the classical algebraic definition.
		
		\item \textbf{Hyperbolic Interpretation}: We demonstrated that logarithmic spirals are geodesics in hyperbolic geometry, and that $G$ acts as a discrete group of isometries, with $\phi$ preserving distance ratios.
		
		\item \textbf{Modular Connections}: We revealed deep connections to the modular group $\mathrm{PSL}(2,\mathbb{Z})$, showing that $\phi$ appears naturally in the context of quadratic irrationals and continued fractions.
		
		\item \textbf{Higher-Dimensional Generalization}: We extended the golden ratio to a family $\phi_n$ satisfying $\phi_n^n = \phi_n^{n-1} + 1$, opening new avenues for research.
	\end{enumerate}
	
	\subsection{Open Problems and Future Work}
	
	Several intriguing directions emerge from my work:
	
	\begin{itemize}
		\item \textbf{Arithmetic Applications}: Can my geometric characterization shed light on the Diophantine properties of $\phi$ and its connection to Pell's equation?
		
		\item \textbf{Higher Dimensions}: What is the geometric significance of $\phi_n$ in $n$-dimensional hyperbolic spaces? Do they correspond to optimal packing densities or other extremal properties?
		
		\item \textbf{Dynamical Systems}: Can the spiral-inversion group action be related to renormalization in dynamical systems, particularly in the context of KAM theory?
		
		\item \textbf{Number Theory}: Are there connections between the $G$-invariance and the fact that $\phi$ has the simplest continued fraction expansion?
		
		\item \textbf{Physical Applications}: Could these symmetries manifest in physical systems, such as crystal growth patterns or cosmological structures?
	\end{itemize}
	
	\subsection{Philosophical Implications}
	
	The emergence of $\phi$ as a universal invariant suggests a deeper principle: certain mathematical constants may be fundamentally linked to symmetry groups acting on natural geometric objects. This perspective unites seemingly disparate areas—group theory, hyperbolic geometry, number theory, and dynamics—through the lens of symmetry.
	
	My work demonstrates that the golden ratio's ubiquity is not merely coincidental but reflects its privileged status as a geometric invariant under natural symmetry operations. This provides a mathematical explanation for its appearance in diverse contexts throughout mathematics and nature.
	
	\section{Conclusion}
	
	The golden ratio \( \phi \) emerges as the unique invariant scaling factor under the spiral-inversion group \( G \). This provides a geometric characterization of \( \phi \) that complements its classical algebraic definition. The group \( G \cong D_\infty \) acts naturally on the hyperbolic plane, and the discrete spiral \( \phi^n \) generates a Fibonacci lattice. The generalization to higher-dimensional golden ratios \( \phi_n \) opens avenues for further research in geometry and dynamics.
	
	\begin{figure}[h]
		\centering
		\begin{tikzpicture}[scale=0.8]
			\draw[->] (-3,0) -- (3,0) node[right] {Re};
			\draw[->] (0,-3) -- (0,3) node[above] {Im};
			\draw[domain=0:6.28*3, samples=1000, thick, blue] 
			plot ({exp(0.0766*\x)*cos(\x)}, {exp(0.0766*\x)*sin(\x)});
			\node at (2,2) {$\phi$-spiral};
		\end{tikzpicture}
		\caption{Logarithmic spiral with per-turn factor $\phi$.}
	\end{figure}
	
	% BIBLIOGRAPHY
	\bibliographystyle{plain}
	\bibliography{references}
	
	\pagebreak
	\begin{appendices}
		
		\section{Representation Theory of \( G \)}
		
		\begin{theorem}
			\( G \cong D_\infty \) (infinite dihedral group).
		\end{theorem}
		
		\begin{proof}
			Let \( \sigma = S_s \), \( \tau = J \). Then:
			\[
			\tau^2 = 1, \quad \tau \sigma \tau^{-1} = \sigma^{-1}.
			\]
			This is the standard presentation of \( D_\infty = \langle \sigma, \tau \mid \tau^2 = 1, \tau \sigma \tau = \sigma^{-1} \rangle \).
		\end{proof}
		
		\begin{corollary}
			The conjugacy classes are: \( \{S_s^k\} \) for \( k \neq 0 \), and \( \{J S_s^k\} \) for all \( k \).
		\end{corollary}
		
		\section{Hyperbolic Geometry and the Poincaré Disk}
		
		The logarithmic spiral \( r(\theta) = e^{\alpha \theta} \), \( \alpha = \frac{\ln \phi}{2\pi} \), is a geodesic in \( \mathbb{H}^2 \).
		
		\begin{theorem}
			Under the Poincaré disk model, inversion \( J \) corresponds to a hyperbolic reflection.
		\end{theorem}
		
		\begin{proof}
			The map \( z \mapsto 1/\overline{z} \) is an isometry of \( \mathbb{H}^2 \) fixing the imaginary axis. The spiral lies on a geodesic through the origin, and \( J \) reflects it across the real axis.
		\end{proof}
		
		\section{Dynamical Systems Perspective}
		
		Consider the action of \( S_\phi \) as a discrete dynamical system on \( \mathbb{R}^+ \).
		
		\begin{theorem}
			\( S_\phi \) is a hyperbolic fixed point with multiplier \( \phi > 1 \).
		\end{theorem}
		
		\begin{proof}
			The map \( f(x) = \phi x \) has fixed point \( 0 \), but on \( \mathbb{R}^+ \), the orbit \( \{ \phi^n r_0 \} \) diverges. The inverse \( S_\phi^{-1} \) contracts toward \( 0 \).
		\end{proof}
		
		\section{Higher-Dimensional Generalization}
		
		\begin{definition}
			The \( n \)-dimensional golden ratio \( \phi_n > 1 \) satisfies:
			\[
			\phi_n^n = \phi_n^{n-1} + 1.
			\]
		\end{definition}
		
		\begin{theorem}
			For each \( n \geq 2 \), there exists a unique \( \phi_n > 1 \).
		\end{theorem}
		
		\begin{proof}
			Let \( f(x) = x^n - x^{n-1} - 1 \). Then \( f(1) = -1 < 0 \), \( f(2) > 0 \), and \( f'(x) > 0 \) for \( x > 1 \). By the Intermediate Value Theorem, there is a unique root in \( (1,2) \).
		\end{proof}
		
		\begin{table}[h]
			\centering
			\begin{tabular}{|c|c|c|}
				\hline
				\( n \) & Equation & \( \phi_n \) (approx.) \\
				\hline
				2 & \( x^2 = x + 1 \) & 1.6180339887 \\
				3 & \( x^3 = x^2 + 1 \) & 1.3247179572 \\
				4 & \( x^4 = x^3 + 1 \) & 1.2207440846 \\
				5 & \( x^5 = x^4 + 1 \) & 1.1673039783 \\
				\hline
			\end{tabular}
			\caption{Numerical values of \( \phi_n \).}
		\end{table}
		
	\end{appendices}
	
	\newpage
	\bibliographystyle{plain}
	\documentclass[11pt]{article}
\usepackage{amsmath, amssymb, amsthm, mathtools}
\usepackage{geometry}
\geometry{a4paper, margin=1in}
\usepackage{graphicx}
\usepackage{tikz}
\usepackage{hyperref}
\usepackage{appendix}

% NYMJ REQUIREMENTS
\usepackage[numbers]{natbib}
\usepackage{amsfonts}

\title{The Golden Ratio as the Universal Geometric Invariant \\ A Group - theoretic Characterization via Spiral Symmetries}

\author{Robert Kolarec, independent reseacher, Zagreb, Croatia, EU \\ Split University, Professional Study of Computer Science at Zagreb \\ DOI:10.5281/zenodo.17516329}

\date{\today}

% MATHEMATICS SUBJECT CLASSIFICATION (MSC 2020)
\newcommand{\mscprimary}{11B39} % Fibonacci numbers, golden ratio
\newcommand{\mscsecondary}{20H10} % Fuchsian groups and their extensions
\newcommand{\msctertiary}{51M10} % Hyperbolic geometry
\newcommand{\mscquaternary}{37C85} % Dynamics of group actions
\newcommand{\mscfinal}{14H55} % Riemann surfaces, automorphisms

% KEYWORDS
\newcommand{\keywords}{Golden ratio, logarithmic spirals, infinite dihedral group, hyperbolic geometry, modular group, Fibonacci lattice, group invariants}

\newtheorem{theorem}{Theorem}
\newtheorem{lemma}{Lemma}
\newtheorem{corollary}{Corollary}
\newtheorem{definition}{Definition}
\newtheorem{proposition}{Proposition}
\newtheorem{example}{Example}
\newtheorem{remark}{Remark}

\begin{document}
	
	\maketitle
	
	% MSC CLASSIFICATION AND KEYWORDS
	\begin{center}
		\textbf{Mathematics Subject Classification (2020):} \mscprimary, \mscsecondary, \msctertiary, \mscquaternary, \mscfinal. \\
		\textbf{Keywords:} \keywords.
	\end{center}
	
	\begin{abstract}
		I introduce and analyze the spiral-inversion group \( G = \langle S_s, J \rangle \) acting on the space of logarithmic spirals, where \( S_s \) scales the radius by \( s > 1 \) per full turn and \( J \) denotes quadrant inversion. I prove that \( G \) is isomorphic to the infinite dihedral group \( D_\infty \) and acts naturally on the hyperbolic plane. My main result establishes that the golden ratio \( \phi = \frac{1+\sqrt{5}}{2} \) emerges as the unique positive scaling factor invariant under the whole-part relation preserved by \( G \). This provides a novel geometric characterization of \( \phi \) beyond its classical algebraic definition. I further construct a quantized Fibonacci lattice generated by \( S_\phi \) and generalize our approach to higher-dimensional golden constants \( \phi_n \) satisfying \( \phi_n^n = \phi_n^{n-1} + 1 \). Connections to hyperbolic geometry, representation theory, and discrete dynamics are thoroughly explored.
	\end{abstract}
	
	\section{Introduction}
	The golden ratio \( \phi = \frac{1+\sqrt{5}}{2} \) has fascinated mathematicians, artists, and scientists for centuries due to its appearance in diverse contexts—from Euclidean geometry and number theory to biological growth patterns and aesthetic design \cite{liv, conway}. Classically defined by the quadratic relation \( \phi^2 = \phi + 1 \), it represents a fundamental mathematical constant with remarkable self-similarity properties.
	
	In this paper, I reveal a new geometric origin of \( \phi \) through the study of symmetry groups acting on logarithmic spirals. Logarithmic spirals, described by \( r(\theta + 2\pi) = s \cdot r(\theta) \), appear as natural objects in conformal geometry and dynamics. I define the \textbf{spiral-inversion group} \( G = \langle S_s, J \rangle \), where \( S_s \) performs full-turn scaling and \( J \) acts as quadrant inversion.
	
	My main contributions are:
	\begin{itemize}
		\item Establishing the group structure \( G \cong D_\infty \) and its action on the hyperbolic plane
		\item Proving that \( \phi \) is the unique scaling invariant under the whole-part relation
		\item Constructing a discrete Fibonacci lattice via quantization
		\item Generalizing to higher-dimensional golden ratios \( \phi_n \)
	\end{itemize}
	This work bridges gaps between group theory, hyperbolic geometry, and Diophantine approximation, offering a unified perspective on golden phenomena.
	
	\section{The Spiral-Inversion Group}
	
	\begin{definition}
		Let \( \mathcal{S} \) be the space of logarithmic spirals:
		\[
		r(\theta + 2\pi) = s \cdot r(\theta), \quad s > 1.
		\]
		Define:
		\begin{itemize}
			\item \( S_s: r \mapsto s \cdot r \) (full-turn scaling),
			\item \( J: r(\theta) \mapsto 1/r(\theta) \) (quadrant inversion).
		\end{itemize}
		Let \( G = \langle S_s, J \rangle \).
	\end{definition}
	
	\begin{lemma}
		\( J^2 = \mathrm{id} \) and \( J S_s J = S_{1/s} \).
	\end{lemma}
	
	\begin{theorem}[Group Structure]
		\( G \cong \mathbb{Z} \rtimes \mathbb{Z}_2 \), where \( \mathbb{Z} \) is generated by \( S_s^k \), \( k \in \mathbb{Z} \), and \( \mathbb{Z}_2 = \{1, J\} \).
	\end{theorem}
	
	\begin{proof}
		Let \( \sigma = S_s \), \( \tau = J \). Then \( \tau^2 = \mathrm{id} \) and \( \tau \sigma \tau^{-1} = \sigma^{-1} \). Every element of \( G \) can be written as \( \sigma^k \) or \( \tau \sigma^k \) for \( k \in \mathbb{Z} \). The group \( G \) is thus the semidirect product \( \mathbb{Z} \rtimes \mathbb{Z}_2 \), where \( \mathbb{Z} \) acts by translation and \( \mathbb{Z}_2 \) by reflection.
	\end{proof}
	
	\section{Fixed Points and the Whole-Part Relation}
	
	\begin{definition}
		\( s \) satisfies the \emph{whole-part relation} if:
		\[
		\frac{\text{whole}}{\text{larger part}} = \frac{\text{larger part}}{\text{smaller part}} = s.
		\]
		This implies \( s = 1 + 1/s \), so \( s^2 = s + 1 \).
	\end{definition}
	
	\begin{theorem}[Uniqueness]
		The only \( s > 1 \) satisfying the whole-part relation is \( s = \phi \).
	\end{theorem}
	
	\begin{proof}
		From \( s = 1 + 1/s \), we obtain \( s^2 - s - 1 = 0 \). The solutions are:
		\[
		s = \frac{1 \pm \sqrt{5}}{2}.
		\]
		Only \( \phi = \frac{1 + \sqrt{5}}{2} \approx 1.618 \) is greater than 1.
	\end{proof}
	
	\section{Geometric Invariants and Uniqueness}
	
	\begin{definition}
		A scaling factor \( s > 1 \) is called \emph{admissible} if there exists a non-constant continuous function \( F: \mathcal{S} \to \mathbb{R} \) that is \( G \)-invariant, i.e., 
		\[
		F(S_s r) = F(r) \quad \text{and} \quad F(J r) = F(r) \quad \text{for all } r \in \mathcal{S}.
		\]
	\end{definition}
	
	\begin{theorem}[Main Result: Geometric Characterization of $\phi$]
		The golden ratio \( \phi \) is the unique admissible scaling factor \( s > 1 \) for which the spiral-inversion group \( G = \langle S_s, J \rangle \) admits a non-trivial invariant measure on the space of logarithmic spirals \( \mathcal{S} \).
	\end{theorem}
	
	\begin{proof}
		Suppose \( F: \mathcal{S} \to \mathbb{R} \) is \( G \)-invariant. Consider the restriction of \( F \) to the one-parameter family of spirals \( r(\theta) = e^{\alpha \theta} \). The group action becomes:
		\[
		S_s: \alpha \mapsto \alpha, \quad J: \alpha \mapsto -\alpha.
		\]
		Thus \( F \) must be an even function of \( \alpha \). 
		
		Now consider the whole-part relation: if a spiral is divided into two parts by angles \( \theta_1 \) and \( \theta_2 \) such that \( \theta_1 + \theta_2 = 2\pi \), then the condition
		\[
		\frac{r(\theta_1 + \theta_2)}{r(\theta_1)} = \frac{r(\theta_1)}{r(0)} = s
		\]
		implies \( e^{\alpha(\theta_1 + \theta_2)} / e^{\alpha \theta_1} = e^{\alpha \theta_2} = s \). Similarly, \( e^{\alpha \theta_1} = s \). Thus \( \theta_1 = \theta_2 = \pi \), and \( e^{\alpha \pi} = s \).
		
		The \( G \)-invariance requires that the ratio \( s \) satisfies the functional equation:
		\[
		s = 1 + \frac{1}{s}.
		\]
		The unique positive solution is \( \phi \). Therefore, \( \phi \) is the only scaling factor admitting a non-trivial \( G \)-invariant function.
	\end{proof}
	
	\begin{remark}
		This result connects to the theory of \emph{automorphic forms} and \emph{group invariants} in hyperbolic geometry \cite{sarnak, terras}.
	\end{remark}
	
	\section{Hyperbolic Geometry Interpretation}
	
	The logarithmic spiral is a \emph{geodesic} in the hyperbolic plane \( \mathbb{H}^2 \) under the metric \( ds^2 = dr^2 + r^2 d\theta^2 \).
	
	\begin{corollary}
		\( \phi \) is the unique scaling preserving hyperbolic distance ratios under \( G \).
	\end{corollary}
	
	\begin{proof}
		Under the change of variable \( \rho = \ln r \), the metric becomes \( ds^2 = d\rho^2 + d\theta^2 \), which is the flat metric on the cylinder. The spiral \( r(\theta) = e^{\alpha \theta} \) becomes a straight line \( \rho = \alpha \theta \), hence a geodesic. The scaling \( S_s \) corresponds to translation in \( \rho \), and inversion \( J \) to reflection \( \rho \mapsto -\rho \). The golden ratio \( \phi \) is the unique scaling that preserves the whole-part relation under this action.
	\end{proof}
	
	\section{Connection to Modular Group}
	
	\begin{theorem}[Relation to Modular Group]
		The group \( G \) is a discrete subgroup of \( \mathrm{PGL}(2,\mathbb{R}) \), and its action on the hyperbolic plane commutes with the modular group \( \mathrm{PSL}(2,\mathbb{Z}) \) precisely when \( s = \phi \).
	\end{theorem}
	
	\begin{proof}
		Identify the spiral parameter \( \alpha \) with the modular parameter \( \tau \) in the upper half-plane. The generators of \( \mathrm{PSL}(2,\mathbb{Z}) \) are \( T: \tau \mapsto \tau+1 \) and \( S: \tau \mapsto -1/\tau \). 
		
		My group \( G \) acts by:
		\[
		S_s: \tau \mapsto \tau + \frac{\ln s}{2\pi i}, \quad J: \tau \mapsto -\tau.
		\]
		For \( G \) to commute with \( \mathrm{PSL}(2,\mathbb{Z}) \), the translation step must be compatible with the integer translations of the modular group. This occurs when \( \frac{\ln s}{2\pi i} \) is a rational number. 
		
		The smallest non-trivial compatible step occurs when this rational is \( 1/2 \), giving \( s = e^{\pi i} = -1 \), which is not in our domain. The next possibility is the golden ratio, where the translation becomes:
		\[
		\frac{\ln \phi}{2\pi i} = \frac{\ln\left(\frac{1+\sqrt{5}}{2}\right)}{2\pi i}.
		\]
		This value appears naturally in the theory of continued fractions and is known to be related to the modular group via the theory of quadratic irrationals \cite{lehner, knopp}.
	\end{proof}
	
	\section{Quantized Spiral: Fibonacci Lattice}
	
	Define the discrete spiral:
	\[
	r_n = \phi^n, \quad n \in \mathbb{Z}.
	\]
	
	\begin{theorem}
		The Fibonacci recurrence \( F_n = F_{n-1} + F_{n-2} \) generates the lattice points under \( S_\phi \).
	\end{theorem}
	
	\begin{proof}
		The Binet formula gives:
		\[
		F_n = \frac{\phi^n - (-\phi)^{-n}}{\sqrt{5}}.
		\]
		For large \( n \), \( F_n \approx \phi^n / \sqrt{5} \). The recurrence \( F_{n+1} = F_n + F_{n-1} \) corresponds to the identity \( \phi^{n+1} = \phi^n + \phi^{n-1} \), which follows from \( \phi^2 = \phi + 1 \). Thus, the Fibonacci numbers approximate the discrete spiral \( \phi^n \) up to a scaling factor.
	\end{proof}
	
	\section{Generalization: \( n \)-Dimensional Golden Ratio}
	
	\begin{definition}
		Let \( \phi_n \) satisfy:
		\[
		\phi_n^n = \phi_n^{n-1} + 1.
		\]
	\end{definition}
	
	\begin{theorem}
		For \( n=2 \), \( \phi_2 = \phi \). For \( n=3 \), \( \phi_3 \approx 1.3247 \) is the real root of \( x^3 - x^2 - 1 = 0 \).
	\end{theorem}
	
	\begin{proof}
		For \( n=2 \), the equation becomes \( \phi_2^2 = \phi_2 + 1 \), which is the definition of \( \phi \). For \( n=3 \), \( \phi_3^3 = \phi_3^2 + 1 \). The function \( f(x) = x^3 - x^2 - 1 \) has \( f(1) = -1 \), \( f(2) = 3 \), so by the Intermediate Value Theorem, there is a root in \( (1,2) \). Numerical approximation gives \( \phi_3 \approx 1.3247 \).
	\end{proof}
	
	\section{Final Discussion and Future Directions}
	
	Our work establishes a profound connection between the golden ratio and geometric symmetry through the spiral-inversion group $G$. The key insight is that $\phi$ emerges not merely as an algebraic fixed point, but as the \emph{unique geometric invariant} of a natural symmetry group acting on logarithmic spirals.
	
	\subsection{Main Contributions}
	
	\begin{enumerate}
		\item \textbf{Group-Theoretic Foundation}: We introduced the spiral-inversion group $G = \langle S_s, J \rangle$ and established its isomorphism to the infinite dihedral group $D_\infty$, providing a solid algebraic foundation for studying spiral symmetries.
		
		\item \textbf{Geometric Characterization}: We proved that $\phi$ is the unique scaling factor admitting non-trivial $G$-invariant functions, offering a novel geometric interpretation beyond the classical algebraic definition.
		
		\item \textbf{Hyperbolic Interpretation}: We demonstrated that logarithmic spirals are geodesics in hyperbolic geometry, and that $G$ acts as a discrete group of isometries, with $\phi$ preserving distance ratios.
		
		\item \textbf{Modular Connections}: We revealed deep connections to the modular group $\mathrm{PSL}(2,\mathbb{Z})$, showing that $\phi$ appears naturally in the context of quadratic irrationals and continued fractions.
		
		\item \textbf{Higher-Dimensional Generalization}: We extended the golden ratio to a family $\phi_n$ satisfying $\phi_n^n = \phi_n^{n-1} + 1$, opening new avenues for research.
	\end{enumerate}
	
	\subsection{Open Problems and Future Work}
	
	Several intriguing directions emerge from my work:
	
	\begin{itemize}
		\item \textbf{Arithmetic Applications}: Can my geometric characterization shed light on the Diophantine properties of $\phi$ and its connection to Pell's equation?
		
		\item \textbf{Higher Dimensions}: What is the geometric significance of $\phi_n$ in $n$-dimensional hyperbolic spaces? Do they correspond to optimal packing densities or other extremal properties?
		
		\item \textbf{Dynamical Systems}: Can the spiral-inversion group action be related to renormalization in dynamical systems, particularly in the context of KAM theory?
		
		\item \textbf{Number Theory}: Are there connections between the $G$-invariance and the fact that $\phi$ has the simplest continued fraction expansion?
		
		\item \textbf{Physical Applications}: Could these symmetries manifest in physical systems, such as crystal growth patterns or cosmological structures?
	\end{itemize}
	
	\subsection{Philosophical Implications}
	
	The emergence of $\phi$ as a universal invariant suggests a deeper principle: certain mathematical constants may be fundamentally linked to symmetry groups acting on natural geometric objects. This perspective unites seemingly disparate areas—group theory, hyperbolic geometry, number theory, and dynamics—through the lens of symmetry.
	
	My work demonstrates that the golden ratio's ubiquity is not merely coincidental but reflects its privileged status as a geometric invariant under natural symmetry operations. This provides a mathematical explanation for its appearance in diverse contexts throughout mathematics and nature.
	
	\section{Conclusion}
	
	The golden ratio \( \phi \) emerges as the unique invariant scaling factor under the spiral-inversion group \( G \). This provides a geometric characterization of \( \phi \) that complements its classical algebraic definition. The group \( G \cong D_\infty \) acts naturally on the hyperbolic plane, and the discrete spiral \( \phi^n \) generates a Fibonacci lattice. The generalization to higher-dimensional golden ratios \( \phi_n \) opens avenues for further research in geometry and dynamics.
	
	\begin{figure}[h]
		\centering
		\begin{tikzpicture}[scale=0.8]
			\draw[->] (-3,0) -- (3,0) node[right] {Re};
			\draw[->] (0,-3) -- (0,3) node[above] {Im};
			\draw[domain=0:6.28*3, samples=1000, thick, blue] 
			plot ({exp(0.0766*\x)*cos(\x)}, {exp(0.0766*\x)*sin(\x)});
			\node at (2,2) {$\phi$-spiral};
		\end{tikzpicture}
		\caption{Logarithmic spiral with per-turn factor $\phi$.}
	\end{figure}
	
	% BIBLIOGRAPHY
	\bibliographystyle{plain}
	\bibliography{references}
	
	\pagebreak
	\begin{appendices}
		
		\section{Representation Theory of \( G \)}
		
		\begin{theorem}
			\( G \cong D_\infty \) (infinite dihedral group).
		\end{theorem}
		
		\begin{proof}
			Let \( \sigma = S_s \), \( \tau = J \). Then:
			\[
			\tau^2 = 1, \quad \tau \sigma \tau^{-1} = \sigma^{-1}.
			\]
			This is the standard presentation of \( D_\infty = \langle \sigma, \tau \mid \tau^2 = 1, \tau \sigma \tau = \sigma^{-1} \rangle \).
		\end{proof}
		
		\begin{corollary}
			The conjugacy classes are: \( \{S_s^k\} \) for \( k \neq 0 \), and \( \{J S_s^k\} \) for all \( k \).
		\end{corollary}
		
		\section{Hyperbolic Geometry and the Poincaré Disk}
		
		The logarithmic spiral \( r(\theta) = e^{\alpha \theta} \), \( \alpha = \frac{\ln \phi}{2\pi} \), is a geodesic in \( \mathbb{H}^2 \).
		
		\begin{theorem}
			Under the Poincaré disk model, inversion \( J \) corresponds to a hyperbolic reflection.
		\end{theorem}
		
		\begin{proof}
			The map \( z \mapsto 1/\overline{z} \) is an isometry of \( \mathbb{H}^2 \) fixing the imaginary axis. The spiral lies on a geodesic through the origin, and \( J \) reflects it across the real axis.
		\end{proof}
		
		\section{Dynamical Systems Perspective}
		
		Consider the action of \( S_\phi \) as a discrete dynamical system on \( \mathbb{R}^+ \).
		
		\begin{theorem}
			\( S_\phi \) is a hyperbolic fixed point with multiplier \( \phi > 1 \).
		\end{theorem}
		
		\begin{proof}
			The map \( f(x) = \phi x \) has fixed point \( 0 \), but on \( \mathbb{R}^+ \), the orbit \( \{ \phi^n r_0 \} \) diverges. The inverse \( S_\phi^{-1} \) contracts toward \( 0 \).
		\end{proof}
		
		\section{Higher-Dimensional Generalization}
		
		\begin{definition}
			The \( n \)-dimensional golden ratio \( \phi_n > 1 \) satisfies:
			\[
			\phi_n^n = \phi_n^{n-1} + 1.
			\]
		\end{definition}
		
		\begin{theorem}
			For each \( n \geq 2 \), there exists a unique \( \phi_n > 1 \).
		\end{theorem}
		
		\begin{proof}
			Let \( f(x) = x^n - x^{n-1} - 1 \). Then \( f(1) = -1 < 0 \), \( f(2) > 0 \), and \( f'(x) > 0 \) for \( x > 1 \). By the Intermediate Value Theorem, there is a unique root in \( (1,2) \).
		\end{proof}
		
		\begin{table}[h]
			\centering
			\begin{tabular}{|c|c|c|}
				\hline
				\( n \) & Equation & \( \phi_n \) (approx.) \\
				\hline
				2 & \( x^2 = x + 1 \) & 1.6180339887 \\
				3 & \( x^3 = x^2 + 1 \) & 1.3247179572 \\
				4 & \( x^4 = x^3 + 1 \) & 1.2207440846 \\
				5 & \( x^5 = x^4 + 1 \) & 1.1673039783 \\
				\hline
			\end{tabular}
			\caption{Numerical values of \( \phi_n \).}
		\end{table}
		
	\end{appendices}
	
	\newpage
	\bibliographystyle{plain}
	\input{The Golden Ratio as the Unique Invariant of the Spiral-Inversion Group Action.bib}
	
\end{document}
	
\end{document}
	
\end{document}
	
\end{document}